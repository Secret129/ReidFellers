\documentclass[12pt]{article}
\usepackage{graphicx}
\usepackage{siunitx}
\begin{document}
\title{Aqueous Reactions}
\author{Reid Fellers}
\date{10/8/2018}
\maketitle

\section{Methods}
\label{sec:methods}

1. A 2 ml sample of of each known compound was obtained.
2. A dropper bottle of each test reagent was obtained.
3. Each known compound (NaCl, Na2CO3, MgSO4, NH4Cl, CoCl2, and FeSO4) was reacted with each test reagent (AgNO3, HCl, and NaOH) and the observations were recorded.
4. A double displacement reaction was carried out
5. two unknown compunds were tested and compared to the known compounds to be identified
\begin {equation}
 \label {eq:1}
 NaCl + AgNO3 = AgCl + NaNO3
 \end {equation}
 \begin {figure}[ht]
   \centering
   \includegraphics[width=0.5\textwidth]{precipitate.jpg}
   \caption{precipitate of AgCl}
   \label{fig : AgCl precipitate}
   \end{figure}
\section{Results}
\label{sec:results}
When tested and compared, unknown one was found to be NaCl. When reacted with AgNO3, both unknown 1 and MgSO4 formed a small white precipitate and turned slightly cloudy when reaccted with HCl and NaOH.

When tested and compared, unknown two was found to be Na2CO3. When reacted with AgNO3, both produced a pearl white precipitate. When reacted with HCl and NaOH, both produced a non reaction.

\end{document}
